\section{A Kernel-PCA framework for normals}\label{sec:kernel-pca-for-normals}
Computing principal components on a subspace of normals is non-trivial due to the fact that normals exist as points lying on the surface of a 2-sphere. For this reason, it is claimed that linear statistical analysis techniques such as PCA cannot be performed directly on normals \footnote{In particular because the definition of a mean is not well defined in arbitrary dimensional spheres}. In order to alleviate this problem, mapping techniques from the unit sphere to an approximate Euclidean space have been proposed \cite{RefWorks:86,RefWorks:90,RefWorks:100}. The most popular proposed techniques are the Azimuthal Equidistant Projection (AEP) and Principal Geodesic Analysis (PGA).  However, in KPCA, we only need to define a kernel that provides an inner product between two vectors in a space. Following the properties of normals described in Section~\ref{sec:properties-of-normals}, we derive kernels for the existing AEP and PGA techniques and show the connection between the angular difference between normals and robust kernels.

We maintain the notation outlined in Section~\ref{subsec:notation-normals} when defining our kernels. Once a vector, $\boldsymbol{x}_k$, has been mapped in to the feature space, we refer to the concatenated feature vectors as $\boldsymbol{v}_k$. The vector $\boldsymbol{v}_k$ will have as many components as the feature space requires.
%%%%%%%%%%%%%%%%%%%%%%%%%%
\subsection{Kernel PCA}\label{sec:kpca}
Given a set of, $K$, $F$-dimensional data vectors stacked in a matrix $ \boldsymbol{X} = [\boldsymbol{x}_1, \ldots, \boldsymbol{x}_K] \in \R^F$, we assume the existence of a positive semi-definite kernel function $k(\circ, \circ) : \R^F \times \R^F \rightarrow \R$. Given that $k(\circ, \circ)$ is positive semi-definite we can use it to define the inner product in an arbitrary dimensional Hilbert space, $\hilbert$, which we will call the feature space. There then exists an implicit mapping, $\Phi$, from the input space $\R^F$ to the feature space, $\hilbert$:
%%%%%%%%%%%%%%%%%%%%%%%%%%%%%%%%%%%%%%%%%%%%
\begin{equation}
    \begin{aligned}\label{eq:implicit-map}
        \Phi : \R^F \rightarrow \hilbert, \; \; \boldsymbol{x} \rightarrow \Phi(\boldsymbol{x})
    \end{aligned}
\end{equation}
%%%%%%%%%%%%%%%%%%%%%%%%%%%%%%%%%%%%%%%%%%%%
Due to the often implicit nature of the mapping $\Phi$, we need only the kernel function since $\langle \Phi(\boldsymbol{x}_i), \Phi(\boldsymbol{x}_j) \rangle  = k (\boldsymbol{x}_i, \boldsymbol{x}_j)$, the so-called kernel trick. Now, component analysis within the feature space is equivalent to
%%%%%%%%%%%%%%%%%%%%%%%%%%%%%%%%%%%%%%%%%%%%
\begin{equation}
    \begin{aligned}\label{eq:feature-space-pca}
        \underset{\boldsymbol{U}_\Phi}{\arg\max} \; \boldsymbol{U}_\Phi^T \bar{\boldsymbol{X}}_\Phi \bar{\boldsymbol{X}}_\Phi^T \boldsymbol{U}_\Phi \qquad \text{s.t.} \; \boldsymbol{U}_\Phi^T \boldsymbol{U}_\Phi = \boldsymbol{I}
    \end{aligned}
\end{equation}
%%%%%%%%%%%%%%%%%%%%%%%%%%%%%%%%%%%%%%%%%%%%
where $\boldsymbol{U}_\Phi = [\boldsymbol{U}_\Phi^1, ..., \boldsymbol{U}_\Phi^P] \in \hilbert$, $\boldsymbol{m}_\Phi = \frac{1}{K} \sum \limits_{i=1}^K \Phi(\boldsymbol{x_i})$ and $\bar{\boldsymbol{X}}_\Phi = [\Phi(\boldsymbol{x_i}) - \boldsymbol{m}_\Phi, ..., \Phi(\boldsymbol{x_K}) - \boldsymbol{m}_\Phi]$.

By noting that $\boldsymbol{\bar{X}}_\Phi \boldsymbol{\bar{X}}_\Phi^T = (\boldsymbol{X}_\Phi \boldsymbol{M}) (\boldsymbol{X}_\Phi \boldsymbol{M})^T$, where $\boldsymbol{M} = \boldsymbol{I} - \frac{1}{K} \boldsymbol{1} \boldsymbol{1}^T$ and $\boldsymbol{1}$ represents a vector of ones, we can find $\boldsymbol{U}_\Phi$ by performing eigenanalysis on $\bar{\boldsymbol{X}}_\Phi^T \bar{\boldsymbol{X}}_\Phi$. Therefore,
%%%%%%%%%%%%%%%%%%%%%%%%%%%%%%%%%%%%%%%%%%%%
\begin{equation}
    \begin{aligned}\label{eq:x-bar-corr}
        \boldsymbol{\bar{X}}_\Phi^T \boldsymbol{\bar{X}}_\Phi = \boldsymbol{V} \boldsymbol{\Lambda} \boldsymbol{V}^T \boldsymbol{U}_{\Phi} &= \boldsymbol{\bar{X}}_\Phi^T \boldsymbol{V} \boldsymbol{\Lambda}^{-\frac{1}{2}}
    \end{aligned}
\end{equation}
%%%%%%%%%%%%%%%%%%%%%%%%%%%%%%%%%%%%%%%%%%%%
Though $\boldsymbol{U}_\Phi$ can be defined, it cannot be calculated explicitly. However, we can compute the KPCA-transformed feature vector $\boldsymbol{\tilde{y}} = [\boldsymbol{y}_1, ..., \boldsymbol{y}_K]$ by:
%%%%%%%%%%%%%%%%%%%%%%%%%%%%%%%%%%%%%%%%%%%%
\begin{equation}
    \begin{aligned}\label{eq:projections}
        \boldsymbol{\tilde{y}} = \boldsymbol{U}_\Phi^T \Phi(\boldsymbol{y}) &= \boldsymbol{\Lambda}^{-\frac{1}{2}} \boldsymbol{V}^T \boldsymbol{\bar{X}}_\Phi^T \Phi(\boldsymbol{y}) \\
        &= \boldsymbol{\Lambda}^{-\frac{1}{2}} \boldsymbol{V}^T \boldsymbol{M} \boldsymbol{X}_\Phi^T \Phi(\boldsymbol{y})
    \end{aligned}
\end{equation}
%%%%%%%%%%%%%%%%%%%%%%%%%%%%%%%%%%%%%%%%%%%%
We can, therefore, define the projection in terms of the kernel function
%%%%%%%%%%%%%%%%%%%%%%%%%%%%%%%%%%%%%%%%%%%%
\begin{equation}\label{eq:kernel-vector}
        \boldsymbol{X}_\Phi^T \Phi(\boldsymbol{y}) = \left[ k(\boldsymbol{y}_1, \boldsymbol{x}_1), \ldots, k(\boldsymbol{y}_K, \boldsymbol{x}_K) \right]^T
\end{equation}
%%%%%%%%%%%%%%%%%%%%%%%%%%%%%%%%%%%%%%%%%%%%
Reconstruction of a vector can be performed by
%%%%%%%%%%%%%%%%%%%%%%%%%%%%%%%%%%%%%%%%%%%%
\begin{equation}\label{eq:vector-reconstruction}
        \boldsymbol{\tilde{X}} = {\Phi}^{-1} \left( \boldsymbol{U}_{\Phi} {\boldsymbol{U}_{\Phi}}^T (\Phi(\boldsymbol{x}) - \boldsymbol{m}_{\Phi}) + \boldsymbol{m}_{\Phi} \right)
\end{equation}
%%%%%%%%%%%%%%%%%%%%%%%%%%%%%%%%%%%%%%%%%%%%
Unfortunately, since ${\Phi}^{-1}$ rarely exists or is extremely expensive to compute, performing reconstruction using (\ref{eq:vector-reconstruction}) is not generally feasible. In these cases, reconstruction can be performed by means of pre-images \cite{RefWorks:254}. However, in the case of the kernels we propose for normals, ${\Phi}^{-1}$ does exist and explicit mapping between the space of normals and kernel space is performed. Finally, we should note here that in the general KPCA framework it is not necessary to subtract the mean. In this case, KPCA can be seen in the perspective of metric multi-dimensional scaling \cite{RefWorks:253}.
%%%%%%%%%%%%%%%%%%%%%%%%%%
%%%%%%%%%%%%%%%%%%%%%%%%%%%%%%%%%%%%%%%%%%%%%%%%%%%%%%%%%%%%%%%%%%%%%%%%%%%%%%%%%%%%%%%%
\subsection{Inner Product Kernel}\label{subsec:ip-kernel}
%%%%%%%%%%%%%%%%%%%%%%%%%%%%%%%%%%%%%%%%%%%%%%%%%%%%%%%%%%%%%%%%%%%%%%%%%%%%%%%%%%%%%%%%
Given that the Euclidean inner product is well defined for normals, as described in (\ref{eq:3d-inner-product}), we can define a kernel of the form
%%%%%%%%%%%%%%%%%%%%%%%%%%%%%%%%%%%%%%%%%%%%
\begin{equation}\label{eq:ip-cosine-kernel}
    k(\boldsymbol{x}_i, \boldsymbol{x}_j) = \sum^K_k {\boldsymbol{n}_k^i}^T \boldsymbol{n}_k^j = \sum^K_k \cos \alpha^{ij}_k
\end{equation}
%%%%%%%%%%%%%%%%%%%%%%%%%%%%%%%%%%%%%%%%%%%%
where $\alpha^{ij}_k = \langle \boldsymbol{n}^i_k, \boldsymbol{n}^j_k \rangle$.

Subtracting the mean would affect the calculation of the cosine and thus would not preserve the cosine distance. Therefore, we note that the inner product mapping is equivalent to performing PCA without subtracting the mean. We refer to this kernel as the inner product (IP) kernel, and denote it as:
%%%%%%%%%%%%%%%%%%%%%%%%%%%%%%%%%%%%%%%%%%%%
\begin{equation}\label{eq:ip-kernel}
    \ip = \boldsymbol{x}_k
\end{equation}
%%%%%%%%%%%%%%%%%%%%%%%%%%%%%%%%%%%%%%%%%%%%
We can explicitly define the inverse mapping for the inner product as the normalisation of each individual normal within the feature space vector, $\boldsymbol{v}_k$:
%%%%%%%%%%%%%%%%%%%%%%%%%%%%%%%%%%%%%%%%%%%%
\begin{equation}\label{eq:inv-ip-kernel}
    \invip = \left[ x_k^1, y_k^1, z_k^1, \ldots, x_k^N, y_k^N, z_k^N \right]^T
\end{equation}
%%%%%%%%%%%%%%%%%%%%%%%%%%%%%%%%%%%%%%%%%%%%

After computing $\ip$, we estimate $\boldsymbol{U}_{IP}$ from (\ref{eq:feature-space-pca}) and set $\boldsymbol{M} = \boldsymbol{I}$. Reconstruction of a test vector of normals $\boldsymbol{x}$ is performed via
%%%%%%%%%%%%%%%%%%%%%%%%%%%%%%%%%%%%%%%%%%%%
\begin{equation}\label{eq:ip-reconstruction}
   \tilde{\boldsymbol{x}} = {\Phi_{IP}}^{-1} \left( \boldsymbol{U}_{IP} {\boldsymbol{U}_{IP}}^T \Phi_{IP}(\boldsymbol{x}) \right)
\end{equation}
%%%%%%%%%%%%%%%%%%%%%%%%%%%%%%%%%%%%%%%%%%%%
%%%%%%%%%%%%%%%%%%%%%%%%%%%%%%%%%%%%%%%%%%%%%%%%%%%%%%%%%%%%%%%%%%%%%%%%%%%%%%%%%%%%%%%%
\subsection{AEP Kernel}\label{subsec:aep-kernel}
%%%%%%%%%%%%%%%%%%%%%%%%%%%%%%%%%%%%%%%%%%%%%%%%%%%%%%%%%%%%%%%%%%%%%%%%%%%%%%%%%%%%%%%%
The azimuthal equidistant projection (AEP) \cite{RefWorks:102,RefWorks:90} is a cartographic projection often used for creating charts centred on the north pole. The projection has the useful property that all lines that pass through the centre of the projection represent geodesics on the surface of a sphere. The projection is constructed at a point $P$ on the surface of a sphere by projecting a local neighbourhood of points around $P$ onto the tangent plane defined at $P$. In terms of normals, we construct the projection by calculating the average normal across the training set at each point, and then projecting each normal on to this tangent plane. This means that the local coordinate system at each point is mean-centred according to the total distribution.

The AEP takes each normal, $\boldsymbol{n}_k^i$ and maps it to a new location on a tangent plane, $\boldsymbol{v}_k^i = [\bar{x}_k^i, \bar{y}_k^i]^T$. The inverse AEP takes the points  $v_k^i$ on the tangent plane and maps them back to normals. For a more detailed derivation of the Azimuthal Equidistant Projection, we invite the reader to consult Smith's paper \cite{RefWorks:90}. Assuming each normal has been projected to its tangent plane according to the AEP function, we define the AEP mapping function as
%%%%%%%%%%%%%%%%%%%%%%%%%%%%%%%%%%%%%%%%%%%%
\begin{equation}\label{eq:aep-kernel}
    \aep = \left[ \bar{x}_k^1, \bar{y}_k^1, \ldots, \bar{x}_k^N, \bar{y}_k^N \right]^T
\end{equation}
%%%%%%%%%%%%%%%%%%%%%%%%%%%%%%%%%%%%%%%%%%%%
and also explicitly define the inverse mapping function
%%%%%%%%%%%%%%%%%%%%%%%%%%%%%%%%%%%%%%%%%%%%
\begin{equation}\label{eq:inv-aep-kernel}
    \invaep = \left[ x_k^1, y_k^1, z_k^1, \dots, x_k^N, y_k^N, z_k^N \right]^T
\end{equation}
%%%%%%%%%%%%%%%%%%%%%%%%%%%%%%%%%%%%%%%%%%%%
After computing $\aep$, we estimate $\boldsymbol{U}_{AEP}$ from (\ref{eq:feature-space-pca}) and set $\boldsymbol{M} = \boldsymbol{I}$. Reconstruction of a test vector of normals $\boldsymbol{x}$ is performed via
%%%%%%%%%%%%%%%%%%%%%%%%%%%%%%%%%%%%%%%%%%%%
\begin{equation}\label{eq:aep-reconstruction}
   \tilde{\boldsymbol{x}} = {\Phi_{AEP}}^{-1} \left( \boldsymbol{U}_{AEP} {\boldsymbol{U}_{AEP}}^T \Phi_{AEP}(\boldsymbol{x}) \right)
\end{equation}
%%%%%%%%%%%%%%%%%%%%%%%%%%%%%%%%%%%%%%%%%%%%
In \cite{RefWorks:90}, $\boldsymbol{U}_{AEP}$ has been used as a prior to perform facial shape-from-shading.
%%%%%%%%%%%%%%%%%%%%%%%%%%%%%%%%%%%%%%%%%%%%%%%%%%%%%%%%%%%%%%%%%%%%%%%%%%%%%%%%%%%%%%%%
\subsection{PGA Kernel}\label{subsec:pga-kernel}
%%%%%%%%%%%%%%%%%%%%%%%%%%%%%%%%%%%%%%%%%%%%%%%%%%%%%%%%%%%%%%%%%%%%%%%%%%%%%%%%%%%%%%%%
Principal geodesic analysis (PGA) \cite{RefWorks:100,RefWorks:86} replaces the linear subspace normally created by PCA by a geodesic manifold. PGA can be used to represent geodesic distances on the surface on any manifold, however, we focus on its use on 2-spheres. This means that every principal component in PGA on 2-spheres represents a great circle. The \textit{extrinsic mean}, as described by Pennec \cite{RefWorks:101}, calculated for PCA does not represent an accurate distance on the manifold. Therefore, we choose to use the \textit{intrinsic mean} defined by the Riemannian distance between two points, $d(\circ,\circ)$. Assuming a set of data points $\boldsymbol{x}$ on embedded on a 2-sphere, $S^2$, we can define the intrinsic mean as $\mu = {\arg\min}_{\boldsymbol{x} \in S^2} \sum_i^K d(\boldsymbol{x}, \boldsymbol{x}_i)$.

Two important operators for the 2-sphere manifold are the logarithmic and exponential maps. Given a point on the surface of a sphere and the normal $\boldsymbol{n}$ at that point, we can define a plane tangent to the sphere at $\boldsymbol{n}$. If we then have a vector $\boldsymbol{v}$, that points to another point on the tangent plane, we can define the exponential map, $Exp_{\boldsymbol{n}}$, as the point on the sphere that is distance $\norm{\boldsymbol{v}}$ along the geodesic in the direction of $\boldsymbol{v}$ from $\boldsymbol{n}$. The logarithmic map, $Log_{\boldsymbol{n}}$ is the inverse of the exponential map. Given a point on the surface of the sphere it returns the corresponding point on the tangent plane at $n$. Given the definition of the logarithmic map, we can define the Riemannian distance for a 2-sphere as $d(\boldsymbol{n}, \boldsymbol{v}) = \norm{Log_{\boldsymbol{n}}(\boldsymbol{v})}$. 

However, as shown by Smith and Hancock in \cite{RefWorks:86}, PGA amounts to performing PCA on the vectors $Log_\mu(\boldsymbol{n}_k)$. Therefore, a kernel-based version of PGA has a mapping function equal to the logarithmic map and an inverse mapping function equal to the exponential map. Assuming we have pre-calculated the intrinsic means, $\mu^i$, we can explicitly define the PGA mapping function as 
%%%%%%%%%%%%%%%%%%%%%%%%%%%%%%%%%%%%%%%%%%%%
\begin{equation}\label{eq:pga-kernel}
    \pga = \left[ Log_{\mu^1}(\boldsymbol{n}_k^1), \ldots, Log_{\mu^N}(\boldsymbol{n}_k^N) \right]^T
\end{equation}
%%%%%%%%%%%%%%%%%%%%%%%%%%%%%%%%%%%%%%%%%%%%
and the inverse mapping as 
%%%%%%%%%%%%%%%%%%%%%%%%%%%%%%%%%%%%%%%%%%%%
\begin{equation}\label{eq:inv-pga-kernel}
    \invpga = \left[ Exp_{\mu^1}(\boldsymbol{v}_k^1), \ldots, Exp_{\mu^N}(\boldsymbol{v}_k^N) \right]^T
\end{equation}
%%%%%%%%%%%%%%%%%%%%%%%%%%%%%%%%%%%%%%%%%%%%
After computing $\pga$, we estimate $\boldsymbol{U}_{PGA}$ from (\ref{eq:feature-space-pca}) and set $\boldsymbol{M} = \boldsymbol{I}$. Reconstruction of a test vector of normals $\boldsymbol{x}$ is performed via
%%%%%%%%%%%%%%%%%%%%%%%%%%%%%%%%%%%%%%%%%%%%
\begin{equation}\label{eq:pga-reconstruction}
   \tilde{\boldsymbol{x}} = {\Phi_{PGA}}^{-1} \left( \boldsymbol{U}_{PGA} {\boldsymbol{U}_{PGA}}^T \Phi_{PGA}(\boldsymbol{x}) \right)
\end{equation}
%%%%%%%%%%%%%%%%%%%%%%%%%%%%%%%%%%%%%%%%%%%%
In \cite{RefWorks:86}, $\boldsymbol{U}_{PGA}$ has been used as a prior to perform facial shape-from-shading.
%%%%%%%%%%%%%%%%%%%%%%%%%%%%%%%%%%%%%%%%%%%%%%%%%%%%%%%%%%%%%%%%%%%%%%%%%%%%%%%%%%%%%%%%
\subsection{Spherical Cosine Kernel}\label{subsec:cosine-kernel}
%%%%%%%%%%%%%%%%%%%%%%%%%%%%%%%%%%%%%%%%%%%%%%%%%%%%%%%%%%%%%%%%%%%%%%%%%%%%%%%%%%%%%%%%
As described in Section~\ref{subsubsec:spherical-normals}, the distance between two normals can also be expressed in terms of spherical coordinates. Motivated by the recent findings on the robustness of the cosine kernel \cite{RefWorks:5, RefWorks:68} we wish to define a cosine-based kernel for use in KPCA. Given the fact that we have two angles, we create a kernel of the form:
%%%%%%%%%%%%%%%%%%%%%%%%%%%%%%%%%%%%%%%%%%%%
\begin{equation}\label{eq:spher-cosine-kernel}
    k(\boldsymbol{x}_i, \boldsymbol{x}_j) = \sum^K_k \cos(\Delta \phi^{ij}_k) + \sum^K_k \cos(\Delta \theta^{ij}_k)
\end{equation}
%%%%%%%%%%%%%%%%%%%%%%%%%%%%%%%%%%%%%%%%%%%%
where $\phi^{ij}_k$ and $\theta^{ij}_k$ are as defined in Equation~\ref{eq:normalised-spherical}. Explicitly, we define the spherical cosine kernel in terms of it's vector components
%%%%%%%%%%%%%%%%%%%%%%%%%%%%%%%%%%%%%%%%%%%%
\begin{equation}
    \begin{aligned}\label{eq:spher-kernel}
        \spher = \left[
                    \tilde{x}_k^1, \tilde{y}_k^1, \tilde{z}_k^1, \sqrt{1 - (\tilde{z}_k^1)^2}, \ldots, \right. \\
                    \left. \tilde{x}_k^N, \tilde{y}_k^N, \tilde{z}_k^N, \sqrt{1 - (\tilde{z}_k^N)^2}
                \right]^T
    \end{aligned}
\end{equation}
%%%%%%%%%%%%%%%%%%%%%%%%%%%%%%%%%%%%%%%%%%%%
The inverse mapping, where we convert from a feature space vector of the form $\boldsymbol{x}_k \in \R^F = [\tilde{x}_k^1, \tilde{y}_k^1, \tilde{z}_k^1, \tilde{sz}_k^1, \ldots]^T$ back to input space, is given as:
%%%%%%%%%%%%%%%%%%%%%%%%%%%%%%%%%%%%%%%%%%%%
\begin{equation}\label{eq:inv-spher-kernel}
    \invspher = \left[ g (\rho_k^1, \psi_k^1), \ldots, g (\rho_k^N, \psi_k^N) \right]^T
\end{equation}
%%%%%%%%%%%%%%%%%%%%%%%%%%%%%%%%%%%%%%%%%%%%
where 
%%%%%%%%%%%%%%%%%%%%%%%%%%%%%%%%%%%%%%%%%%%%
\begin{equation}
    \begin{aligned}\label{eq:inv-spher-g}
        &\rho_k^i = \arctan [ \frac{\tilde{y}_k^i}{\sqrt{(\tilde{x}_k^i)^2 + (\tilde{y}_k^i)^2}} / \frac{\tilde{x}_k^i}{\sqrt{(\tilde{x}_k^i)^2 + (\tilde{y}_k^i)^2}} ] \\
        &\psi_k^i = \arctan [ \frac{\tilde{sz}_k^i}{\sqrt{(\tilde{z}_k^i)^2 + (\tilde{sz}_k^i)^2}} / \frac{\tilde{z}_k^i}{\sqrt{(\tilde{z}_k^i)^2 + (\tilde{sz}_k^i)^2}} ] \\
        &g(\rho_k^i, \psi_k^i) = [\cos \psi_k^i \sin \rho_k^i, \sin \psi_k^i \sin \rho_k^i, \cos \psi_k^i]^T
    \end{aligned}
\end{equation}
%%%%%%%%%%%%%%%%%%%%%%%%%%%%%%%%%%%%%%%%%%%%
After computing $\spher$, we estimate $\boldsymbol{U}_{SPHER}$ from (\ref{eq:feature-space-pca}) and set $\boldsymbol{M} = \boldsymbol{I}$. Reconstruction of a test vector of normals $\boldsymbol{x}$ is performed via
%%%%%%%%%%%%%%%%%%%%%%%%%%%%%%%%%%%%%%%%%%%%
\begin{equation}\label{eq:spher-reconstruction}
   \tilde{\boldsymbol{x}} = {\Phi_{SPHER}}^{-1} \left( \boldsymbol{U}_{SPHER} {\boldsymbol{U}_{SPHER}}^T \Phi_{SPHER}(\boldsymbol{x}) \right)
\end{equation}
%%%%%%%%%%%%%%%%%%%%%%%%%%%%%%%%%%%%%%%%%%%%
%%%%%%%%%%%%%%%%%%%%%%%%%%
%%%%%%%%%%%%%%%%%%%%%%%%%%%%%%%%%%%%%%%%%%%%
\subsection{Geometric Shape-from-shading}\label{subsec:gsfs}
Shape-from-shading (SFS) is the name given to algorithms that attempt to recover a representation of shape from the pixel intensities of an image. The most challenging subset of these algorithms concentrate on recovering shape from an image illuminated by a known single point light source and known reflectance behaviour. The most commonly assumed reflectance behaviour is lambertian reflectance, which assumes the following relationship between the intrinsic shape of an object and the intensity of the light reflected from it
%%%%%%%%%%%%%%%%%%%%%%%%%%%%
\begin{equation}\label{eq:lambertian-reflectance}
    E = \rho (\boldsymbol{n} \cdot \boldsymbol{s})
\end{equation}
%%%%%%%%%%%%%%%%%%%%%%%%%%%%
where $\boldsymbol{n}$ is the normal of the surface at the point of reflectance, $\boldsymbol{s}$ is the light direction and $\rho$ is the albedo. Lambertian reflectance assumes that an object exhibits ideal diffuse reflectivity, and so the albedo term represents a reflectivity coefficient. Even if given the albedo and light direction, (\ref{eq:lambertian-reflectance}) is still under-constrained. Many techniques have been proposed to solve SFS \cite{RefWorks:249, RefWorks:225, RefWorks:270}, and the majority concentrate on recovering some representation of the surface normal, as given by (\ref{eq:lambertian-reflectance}).

Geometric shape-from-shading (GSFS) is the name given by Smith and Hancock to their SFS algorithm for statistical reconstruction of facial needle-maps \cite{RefWorks:86, RefWorks:90}. Although we have chosen to place all KPCA kernels within this algorithm, we would stress that this is merely to provide a practical demonstration of the power of the proposed kernel component analysis. The use of a statistical prior, however, does produce superior results when compared to state-of-the-art SFS techniques that have no prior knowledge of object shape.

GSFS extends the SFS algorithm given by Worthington and Hancock \cite{RefWorks:252} to include a statistical prior on needle-maps. The algorithm is simple to compute and produces results that are visually appealing and guaranteed to represent the space of faces. They initialise the needle-map by assuming global convexity, and then proceed to iterate by first reconstructing the normals using the PCA model and then enforcing the hard-irradiance constraint. The hard-irradiance constraint is enforced by rotating the potentially off-cone reconstructed surface normal back onto the reflectance cone specified by the light direction and intensity of a given pixel. An overview of the algorithm is given in Algorithm~\ref{alg:gsfs}. We augment the original GSFS algorithm by replacing the statistical reconstruction step with each of the previously defined kernels.

To recover a depthmap from the recovered set of normals requires an integration step. However, as discussed in \cite{RefWorks:281}, there are a number of ways that this integration can be performed. We have chosen to use the elegant technique proposed by Frankot and Chellappa \cite{RefWorks:99} due to its efficiency and the quality of its reconstruction.

\alglanguage{pseudocode}
\begin{algorithm}[ht]
    \caption{{\sc Geometric shape-from-shading}}
    \label{alg:gsfs}
        \begin{algorithmic}
            \Statex{Iterate until $\sum_{i,j} \arccos \left({\boldsymbol{n}(i, j)}' \cdot {\boldsymbol{n}(i, j)}'' \right) < \epsilon$:}
            \Statex \hspace{\algorithmicindent} (1) Calculate an initial estimate of the surface normals.
            \Statex \hspace{\algorithmicindent} (2) Project the needle-map into the feature space using one of the kernels defined in Section~\ref{sec:kernel-pca-for-normals}: $\Phi_{F}(\boldsymbol{x})$
            \Statex \hspace{\algorithmicindent} (3) Reconstruct the best fit feature space vector: $\boldsymbol{U}_{F} {\boldsymbol{U}_{F}}^T \Phi_{F}(\boldsymbol{x})$.
            \Statex \hspace{\algorithmicindent} (4) Use the inverse mapping to recreate a set of surface normals, ${x}' = {\Phi_{F}}^{-1} \left( \boldsymbol{U}_{F} {\boldsymbol{U}_{F}}^T \Phi_{F}(\boldsymbol{x}) \right)$, with individual normals ${\boldsymbol{n}(i, j)}'$
            \Statex \hspace{\algorithmicindent} (5) Enforce hard-irradiance constraint on the reconstructed normals to find the on-cone surface normal, ${\boldsymbol{n}(i, j)}''$.
        \end{algorithmic}
\end{algorithm}