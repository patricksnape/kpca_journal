\documentclass[10pt,journal,cspaper,compsoc]{IEEEtran}

\usepackage{graphicx}
\usepackage{amsmath}
\usepackage{amssymb}
\usepackage{amsbsy}
\usepackage{xfrac}
\usepackage{subcaption}
\usepackage{algorithm}
\usepackage{algpseudocode}

\usepackage[breaklinks=true,bookmarks=false]{hyperref}

\newcommand{\abs}[1]{\lvert #1 \rvert}
\newcommand{\norm}[1]{\lVert #1 \rVert}
\newcommand{\sgn}{\operatorname{sgn}}
\newcommand{\argmin}{\operatorname{arg\!min}}
\newcommand{\argmax}{\operatorname{arg\!max}}
\newcommand{\R}{\mathbb{R}}
\newcommand{\hilbert}{\mathcal{H}}

% Inner product mapping
\newcommand{\ip}{\phi_{IP}(\boldsymbol{x}_k)}
% Inverse inner product mapping
\newcommand{\invip}{{\phi_{IP}}^{-1}(\boldsymbol{v}_k)}
% Spherical mapping
\newcommand{\spher}{\phi_{SPHER}(\boldsymbol{x}_k)}
% Inverse Spherical mapping
\newcommand{\invspher}{{\phi_{SPHER}}^{-1}(\boldsymbol{v}_k)}
% AEP mapping
\newcommand{\aep}{\phi_{AEP}(\boldsymbol{x}_k)}
% Inverse AEP mapping
\newcommand{\invaep}{{\phi_{AEP}}^{-1}(\boldsymbol{v}_k)}
% PGA mapping
\newcommand{\pga}{\phi_{PGA}(\boldsymbol{x}_k)}
% Inverse PGA mapping
\newcommand{\invpga}{{\phi_{PGA}}^{-1}(\boldsymbol{v}_k)}
% Least squares mapping
\newcommand{\ls}{\phi_{LS}(\boldsymbol{x}_k)}
% Inverse least squares mapping
\newcommand{\invls}{{\phi_{LS}}^{-1}(\boldsymbol{v}_k)}


% Image commands
\newcommand{\origimg}[1]{
    \begin{subfigure}{0.11\textwidth}
            \centering
            \includegraphics[width=\textwidth]{images/results/photoface/#1.png}
            \label{fig:results-photoface-input-#1}
    \end{subfigure}
}
\newcommand{\origimgtop}[2]{
    \begin{subfigure}{0.11\textwidth}
            \centering
            \caption*{#2}
            \includegraphics[width=\textwidth]{images/results/photoface/#1.png}
            \label{fig:results-photoface-input-#1}
    \end{subfigure}
}
\newcommand{\sfsimgs}[2]{
    \begin{subfigure}{0.11\textwidth}
            \centering
            \includegraphics[width=\textwidth]{images/results/photoface/#1_#2.png}
            \label{fig:results-photoface-#1-#2}
    \end{subfigure}
}
\newcommand{\sfsimgstop}[3]{
    \begin{subfigure}{0.11\textwidth}
            \centering
            \caption*{#2}
            \includegraphics[width=\textwidth]{images/results/photoface/#1_#3.png}
            \label{fig:results-photoface-#1-#3}
    \end{subfigure}
}

\newcommand{\sfsimgsalltop}[1]{
    \origimgtop{#1}{Input Image}
    \sfsimgstop{#1}{Ground Truth}{ground_truth}
    \sfsimgstop{#1}{AEP}{aep}
    \sfsimgstop{#1}{IP}{cosine}
    \sfsimgstop{#1}{PGA}{pga}
    \sfsimgstop{#1}{SPHERICAL}{spherical}
    \sfsimgstop{#1}{SIRFS}{sirfs}
}

\newcommand{\sfsimgsall}[1]{
    \origimg{#1}
    \sfsimgs{#1}{ground_truth}
    \sfsimgs{#1}{aep}
    \sfsimgs{#1}{cosine}
    \sfsimgs{#1}{pga}
    \sfsimgs{#1}{spherical}
    \sfsimgs{#1}{sirfs}
} % Load commands

\begin{document}

\title{Lucas-Kanade on Normals}


\author{Patrick~Snape,~\IEEEmembership{Student~Member,~IEEE,}
        and~Stefanos~Zafeiriou,~\IEEEmembership{Member,~IEEE}}


% The paper headers
%\markboth{IEEE Transactions On Pattern Analysis And Machine Intelligence,~Vol.~1, No.~1, December~2013}{A}


\IEEEcompsoctitleabstractindextext{%
\begin{abstract}
The unwritten abstract.
\end{abstract}

\begin{keywords}
Normals, Active Appearance Models, Shape-From-Shading
\end{keywords}}


\maketitle


% To allow for easy dual compilation without having to reenter the
% abstract/keywords data, the \IEEEcompsoctitleabstractindextext text will
% not be used in maketitle, but will appear (i.e., to be "transported")
% here as \IEEEdisplaynotcompsoctitleabstractindextext when compsoc mode
% is not selected <OR> if conference mode is selected - because compsoc
% conference papers position the abstract like regular (non-compsoc)
% papers do!
\IEEEdisplaynotcompsoctitleabstractindextext
% For peerreview papers, this IEEEtran command inserts a page break and
% creates the second title. It will be ignored for other modes.
\IEEEpeerreviewmaketitle

%\ifCLASSOPTIONcompsoc
%  \noindent\raisebox{2\baselineskip}[0pt][0pt]%
%  {\parbox{\columnwidth}{\section{Introduction}\label{sec:introduction}%
%  \global\everypar=\everypar}}%
%  \vspace{-1\baselineskip}\vspace{-\parskip}\par
%\else
%  \section{Introduction}\label{sec:introduction}\par
%\fi
%
% Admittedly, this is a hack and may well be fragile, but seems to do the
% trick for me. Note the need to keep any \label that may be used right
% after \section in the above as the hack puts \section within a raised box.



% The very first letter is a 2 line initial drop letter followed
% by the rest of the first word in caps (small caps for compsoc).
% 
% form to use if the first word consists of a single letter:
% \IEEEPARstart{A}{demo} file is ....
% 
% form to use if you need the single drop letter followed by
% normal text (unknown if ever used by IEEE):
% \IEEEPARstart{A}{}demo file is ....
% 
% Some journals put the first two words in caps:
% \IEEEPARstart{T}{his demo} file is ....
% 
% Here we have the typical use of a "T" for an initial drop letter
% and "HIS" in caps to complete the first word.

% use section* for acknowledgement
%\ifCLASSOPTIONcompsoc
  % The Computer Society usually uses the plural form
%  \section*{Acknowledgments}
%\else
  % regular IEEE prefers the singular form
%  \section*{Acknowledgment}
%\fi

%Acknowledgment text

\section{Introduction}\label{sec:intro}
%%% Component analysis
Component analysis is an important tool for understanding and processing visual data. Computer vision problems often involve high-dimensional data that are non-linearly related. This has spurred a lot of interest in the development of efficient and effective techniques for computing nonlinear dimensionality reduction \cite{RefWorks:91,RefWorks:92,RefWorks:93,RefWorks:94}. In parallel with this, there has been increased interest in appearance based object recognition and reconstruction \cite{RefWorks:95,RefWorks:96,RefWorks:97,RefWorks:98}. %% Existing component analysis and it's problems
However, much of the existing work on the statistical analysis of appearance-based models has focused on the use of shape or texture, which are not necessarily robust descriptors of an object. Texture, for example, is often corrupted by outliers such as occlusions, cast shadows and illumination changes. 
%% Solution? Normals.
Surface normals, on the other hand, are invariant to changes in illumination and still offer a method for shape recovery via integration \cite{RefWorks:99}. In fact, many reconstruction techniques, such as shape-from-shading (SFS)\cite{RefWorks:230, RefWorks:252, RefWorks:225}, recover normals directly and thus component analysis of normals is beneficial.

% However, normals are non-linear
If we wish to perform subspace analysis on normals, we must consider the properties of normal spaces. A distribution of unit normals define a set of points that lie upon the surface of a spherical manifold. Therefore, the computation of distances between normals is a non-trivial task. In order to perform subspace analysis on manifolds we have to be able to compute non-linear relationships. 
% But we can use KPCA for non-linear relationships
Kernel Principal Component Analysis (KPCA), is a non-linear generalisation of the linear data analysis method Principal Component Analysis (PCA). KPCA is able to perform subspace analysis within arbitrary dimensional Hilbert spaces, including the subspace of normals. By providing a kernel function that defines an inner product within a Hilbert space, we can perform component analysis in spaces where PCA would normally be infeasible. 

% So what do we do about it?
In this paper, we show the power of using KPCA to perform component analysis of normals. The difference of the proposed framework is that instead of using of-the-shelf kernels such as RBF or polynomial kernels used in the majority of KPCA papers, we are interested only in kernels tailored to normals. By defining kernel functions on normals, we allow more robust component analysis to be computed. In particular, we propose a novel kernel based upon the angular difference between normals that is shown to be more robust than any existing descriptor of normals. We also investigate previous work on component analysis of normals, and incorporate it into our framework.

%% Previous work?
Existing work on constructing a feature space whereby distances between normals can be computed has been investigated by Smith and Hancock \cite{RefWorks:90,RefWorks:86}. Smith and Hancock propose two projection methods, the Azimuthal Equidistant Projection (AEP) \cite{RefWorks:102} and Principal Geodesic Analysis (PGA) \cite{RefWorks:100,RefWorks:101}. By projecting normals into tangent spaces, they show that linear component analysis can be performed. Smith and Hancock argue that projection of normals is a requirement for the component analysis of normals. However, although the observation that computing distances between normals is non-trivial is correct, this does not actually prevent component analysis directly on normals (\ie without applying any transformation). By formulating the component analysis in terms of a kernel, it becomes obvious that component analysis \textit{can be performed directly on normals by defining the kernel as the Euclidean inner product}. We generalise AEP and PGA as kernels in our framework and provide a kernel for component analysis directly on normals without transformation.

%% Little work on component analysis of normals!
Other than contributions by Smith and Hancock \cite{RefWorks:90,RefWorks:86}, little work has been done on the component analysis of normals. We are thus most interested in investigating the robustness of the subspace of normals. Although normals may be extracted from any class of objects, our results focus on faces. Despite the lack of research on the subspace of normals, there has been a lot of interest in SFS algorithms \cite{RefWorks:230}. We are not interested in comparing the abilities of different SFS algorithms and use a SFS algorithm proposed by Smith and Hancock merely due to the ease of embedding a statistical model. We have, however, compared against a state-of-the-art SFS algorithm in the form of SIRFS \cite{RefWorks:225} and thus show the value of prior knowledge in SFS algorithms. We also note that Kemelmacher and Basri \cite{RefWorks:226} provide a state-of-the-art shape recovery procedure that focuses on faces. However, they directly recover the shape and thus are subject to restrictive boundary conditions. In particular, their technique requires the boundary of the reference shape to lie upon "slowly changing parts of the face". Statistical models of normals have no such constraint and can recover a much larger portion of the face.

%% Summarise
We summarise our contributions as follows:
%%%%%%%%%%%%%%%%%%
\begin{itemize}

  \item We provide a kernel-based framework for performing statistical component analysis of normals.
  \item We formulate two existing projection operations, the AEP and PGA within our framework.
  \item We show that components \textit{can} be extracted directly from normals, which becomes clear within the KPCA framework.
  \item We provide a novel robust kernel based on the cosine of the angles between normals.
  \item We give quantitative analysis as to the robustness of the kernels and also show SFS results that out-perform existing SFS techniques.

\end{itemize}

\subsection{Kernel PCA}\label{sec:kpca}
Given a set of, $K$, $F$-dimensional data vectors stacked in a matrix $ \boldsymbol{X} = [\boldsymbol{x}_1, \ldots, \boldsymbol{x}_K] \in \R^F$, we assume the existence of a positive semi-definite kernel function $k(\circ, \circ) : \R^F \times \R^F \rightarrow \R$. Given that $k(\circ, \circ)$ is positive semi-definite we can use it to define the inner product in an arbitrary dimensional Hilbert space, $\hilbert$, which we will call the feature space. There then exists an implicit mapping, $\Phi$, from the input space $\R^F$ to the feature space, $\hilbert$:
%%%%%%%%%%%%%%%%%%%%%%%%%%%%%%%%%%%%%%%%%%%%
\begin{equation}
    \begin{aligned}\label{eq:implicit-map}
        \Phi : \R^F \rightarrow \hilbert, \; \; \boldsymbol{x} \rightarrow \Phi(\boldsymbol{x})
    \end{aligned}
\end{equation}
%%%%%%%%%%%%%%%%%%%%%%%%%%%%%%%%%%%%%%%%%%%%
Due to the often implicit nature of the mapping $\Phi$, we need only the kernel function since $\langle \Phi(\boldsymbol{x}_i), \Phi(\boldsymbol{x}_j) \rangle  = k (\boldsymbol{x}_i, \boldsymbol{x}_j)$, the so-called kernel trick. Now, component analysis within the feature space is equivalent to
%%%%%%%%%%%%%%%%%%%%%%%%%%%%%%%%%%%%%%%%%%%%
\begin{equation}
    \begin{aligned}\label{eq:feature-space-pca}
        \underset{\boldsymbol{U}_\Phi}{\arg\max} \; \boldsymbol{U}_\Phi^T \bar{\boldsymbol{X}}_\Phi \bar{\boldsymbol{X}}_\Phi^T \boldsymbol{U}_\Phi \qquad \text{s.t.} \; \boldsymbol{U}_\Phi^T \boldsymbol{U}_\Phi = \boldsymbol{I}
    \end{aligned}
\end{equation}
%%%%%%%%%%%%%%%%%%%%%%%%%%%%%%%%%%%%%%%%%%%%
where $\boldsymbol{U}_\Phi = [\boldsymbol{U}_\Phi^1, ..., \boldsymbol{U}_\Phi^P] \in \hilbert$, $\boldsymbol{m}_\Phi = \frac{1}{K} \sum \limits_{i=1}^K \Phi(\boldsymbol{x_i})$ and $\bar{\boldsymbol{X}}_\Phi = [\Phi(\boldsymbol{x_i}) - \boldsymbol{m}_\Phi, ..., \Phi(\boldsymbol{x_K}) - \boldsymbol{m}_\Phi]$.

By noting that $\boldsymbol{\bar{X}}_\Phi \boldsymbol{\bar{X}}_\Phi^T = (\boldsymbol{X}_\Phi \boldsymbol{M}) (\boldsymbol{X}_\Phi \boldsymbol{M})^T$, where $\boldsymbol{M} = \boldsymbol{I} - \frac{1}{K} \boldsymbol{1} \boldsymbol{1}^T$ and $\boldsymbol{1}$ represents a vector of ones, we can find $\boldsymbol{U}_\Phi$ by performing eigenanalysis on $\bar{\boldsymbol{X}}_\Phi^T \bar{\boldsymbol{X}}_\Phi$. Therefore,
%%%%%%%%%%%%%%%%%%%%%%%%%%%%%%%%%%%%%%%%%%%%
\begin{equation}
    \begin{aligned}\label{eq:x-bar-corr}
        \boldsymbol{\bar{X}}_\Phi^T \boldsymbol{\bar{X}}_\Phi = \boldsymbol{V} \boldsymbol{\Lambda} \boldsymbol{V}^T \boldsymbol{U}_{\Phi} &= \boldsymbol{\bar{X}}_\Phi^T \boldsymbol{V} \boldsymbol{\Lambda}^{-\frac{1}{2}}
    \end{aligned}
\end{equation}
%%%%%%%%%%%%%%%%%%%%%%%%%%%%%%%%%%%%%%%%%%%%
Though $\boldsymbol{U}_\Phi$ can be defined, it cannot be calculated explicitly. However, we can compute the KPCA-transformed feature vector $\boldsymbol{\tilde{y}} = [\boldsymbol{y}_1, ..., \boldsymbol{y}_K]$ by:
%%%%%%%%%%%%%%%%%%%%%%%%%%%%%%%%%%%%%%%%%%%%
\begin{equation}
    \begin{aligned}\label{eq:projections}
        \boldsymbol{\tilde{y}} = \boldsymbol{U}_\Phi^T \Phi(\boldsymbol{y}) &= \boldsymbol{\Lambda}^{-\frac{1}{2}} \boldsymbol{V}^T \boldsymbol{\bar{X}}_\Phi^T \Phi(\boldsymbol{y}) \\
        &= \boldsymbol{\Lambda}^{-\frac{1}{2}} \boldsymbol{V}^T \boldsymbol{M} \boldsymbol{X}_\Phi^T \Phi(\boldsymbol{y})
    \end{aligned}
\end{equation}
%%%%%%%%%%%%%%%%%%%%%%%%%%%%%%%%%%%%%%%%%%%%
We can, therefore, define the projection in terms of the kernel function
%%%%%%%%%%%%%%%%%%%%%%%%%%%%%%%%%%%%%%%%%%%%
\begin{equation}\label{eq:kernel-vector}
        \boldsymbol{X}_\Phi^T \Phi(\boldsymbol{y}) = \left[ k(\boldsymbol{y}_1, \boldsymbol{x}_1), \ldots, k(\boldsymbol{y}_K, \boldsymbol{x}_K) \right]^T
\end{equation}
%%%%%%%%%%%%%%%%%%%%%%%%%%%%%%%%%%%%%%%%%%%%
Reconstruction of a vector can be performed by
%%%%%%%%%%%%%%%%%%%%%%%%%%%%%%%%%%%%%%%%%%%%
\begin{equation}\label{eq:vector-reconstruction}
        \boldsymbol{\tilde{X}} = {\Phi}^{-1} \left( \boldsymbol{U}_{\Phi} {\boldsymbol{U}_{\Phi}}^T (\Phi(\boldsymbol{x}) - \boldsymbol{m}_{\Phi}) + \boldsymbol{m}_{\Phi} \right)
\end{equation}
%%%%%%%%%%%%%%%%%%%%%%%%%%%%%%%%%%%%%%%%%%%%
Unfortunately, since ${\Phi}^{-1}$ rarely exists or is extremely expensive to compute, performing reconstruction using (\ref{eq:vector-reconstruction}) is not generally feasible. In these cases, reconstruction can be performed by means of pre-images \cite{RefWorks:254}. However, in the case of the kernels we propose for normals, ${\Phi}^{-1}$ does exist and explicit mapping between the space of normals and kernel space is performed. Finally, we should note here that in the general KPCA framework it is not necessary to subtract the mean. In this case, KPCA can be seen in the perspective of metric multi-dimensional scaling \cite{RefWorks:253}.
\input{subspace-analysis}
%%%%%%%%%%%%%%%%%%%%%%%%%%%%%%%%%%%%%%%%%%%%
\subsection{Geometric Shape-from-shading}\label{subsec:gsfs}
Shape-from-shading (SFS) is the name given to algorithms that attempt to recover a representation of shape from the pixel intensities of an image. The most challenging subset of these algorithms concentrate on recovering shape from an image illuminated by a known single point light source and known reflectance behaviour. The most commonly assumed reflectance behaviour is lambertian reflectance, which assumes the following relationship between the intrinsic shape of an object and the intensity of the light reflected from it
%%%%%%%%%%%%%%%%%%%%%%%%%%%%
\begin{equation}\label{eq:lambertian-reflectance}
    E = \rho (\boldsymbol{n} \cdot \boldsymbol{s})
\end{equation}
%%%%%%%%%%%%%%%%%%%%%%%%%%%%
where $\boldsymbol{n}$ is the normal of the surface at the point of reflectance, $\boldsymbol{s}$ is the light direction and $\rho$ is the albedo. Lambertian reflectance assumes that an object exhibits ideal diffuse reflectivity, and so the albedo term represents a reflectivity coefficient. Even if given the albedo and light direction, (\ref{eq:lambertian-reflectance}) is still under-constrained. Many techniques have been proposed to solve SFS \cite{RefWorks:249, RefWorks:225, RefWorks:270}, and the majority concentrate on recovering some representation of the surface normal, as given by (\ref{eq:lambertian-reflectance}).

Geometric shape-from-shading (GSFS) is the name given by Smith and Hancock to their SFS algorithm for statistical reconstruction of facial needle-maps \cite{RefWorks:86, RefWorks:90}. Although we have chosen to place all KPCA kernels within this algorithm, we would stress that this is merely to provide a practical demonstration of the power of the proposed kernel component analysis. The use of a statistical prior, however, does produce superior results when compared to state-of-the-art SFS techniques that have no prior knowledge of object shape.

GSFS extends the SFS algorithm given by Worthington and Hancock \cite{RefWorks:252} to include a statistical prior on needle-maps. The algorithm is simple to compute and produces results that are visually appealing and guaranteed to represent the space of faces. They initialise the needle-map by assuming global convexity, and then proceed to iterate by first reconstructing the normals using the PCA model and then enforcing the hard-irradiance constraint. The hard-irradiance constraint is enforced by rotating the potentially off-cone reconstructed surface normal back onto the reflectance cone specified by the light direction and intensity of a given pixel. An overview of the algorithm is given in Algorithm~\ref{alg:gsfs}. We augment the original GSFS algorithm by replacing the statistical reconstruction step with each of the previously defined kernels.

To recover a depthmap from the recovered set of normals requires an integration step. However, as discussed in \cite{RefWorks:281}, there are a number of ways that this integration can be performed. We have chosen to use the elegant technique proposed by Frankot and Chellappa \cite{RefWorks:99} due to its efficiency and the quality of its reconstruction.

\alglanguage{pseudocode}
\begin{algorithm}[ht]
    \caption{{\sc Geometric shape-from-shading}}
    \label{alg:gsfs}
        \begin{algorithmic}
            \Statex{Iterate until $\sum_{i,j} \arccos \left({\boldsymbol{n}(i, j)}' \cdot {\boldsymbol{n}(i, j)}'' \right) < \epsilon$:}
            \Statex \hspace{\algorithmicindent} (1) Calculate an initial estimate of the surface normals.
            \Statex \hspace{\algorithmicindent} (2) Project the needle-map into the feature space using one of the kernels defined in Section~\ref{sec:kernel-pca-for-normals}: $\Phi_{F}(\boldsymbol{x})$
            \Statex \hspace{\algorithmicindent} (3) Reconstruct the best fit feature space vector: $\boldsymbol{U}_{F} {\boldsymbol{U}_{F}}^T \Phi_{F}(\boldsymbol{x})$.
            \Statex \hspace{\algorithmicindent} (4) Use the inverse mapping to recreate a set of surface normals, ${x}' = {\Phi_{F}}^{-1} \left( \boldsymbol{U}_{F} {\boldsymbol{U}_{F}}^T \Phi_{F}(\boldsymbol{x}) \right)$, with individual normals ${\boldsymbol{n}(i, j)}'$
            \Statex \hspace{\algorithmicindent} (5) Enforce hard-irradiance constraint on the reconstructed normals to find the on-cone surface normal, ${\boldsymbol{n}(i, j)}''$.
        \end{algorithmic}
\end{algorithm}
\input{experiments}
\section{Conclusion}\label{sec:conclusion}
We introduced a kernel-based framework for performing component analysis of normals. We linked existing projection methods, the azimuthal equidistant projection and principal geodesic analysis, to a unified framework. We show that, with the help of our kernel-based formulation, component analysis can be performed directly upon normals without transformation. We also propose a new robust kernel for performing component analysis on normals. In particular, our new kernel based on the angular distance, shows qualitative and quantitative improvement over existing techniques in both artificial reconstruction and SFS settings.

% if have a single appendix:
%\appendix[Proof of the Zonklar Equations]
% or
%\appendix  % for no appendix heading
% do not use \section anymore after \appendix, only \section*
% is possibly needed

% use appendices with more than one appendix
% then use \section to start each appendix
% you must declare a \section before using any
% \subsection or using \label (\appendices by itself
% starts a section numbered zero.)


%\input{appendices}

{\small
\bibliographystyle{ieeetr}
\bibliography{egbib}
}

\end{document}
