\section{Introduction}\label{sec:introduction}
%%%%%%%%%%%%%%%%%%%%%%%%%%%%%%%%%%%%%%%%%%%%%%%%%%%%%%%%%%%%%%%%%%%%%%%%%%%%%%%%
Alignment of images is a crucial requirement in a variety of applications in the fields of medical imaging \cite{RefWorks:69,RefWorks:70}, robotics \cite{RefWorks:11}, and face analysis \cite{RefWorks:6}. These alignment algorithms concentrate on recovering a warp that best minimises an error metric between two images. Three-dimensional data is often aligned using algorithms such as the iterative closest point (ICP) \cite{RefWorks:1} algorithm, which have been proven to be very effective. However, ICP involves a costly correspondence step that is not necessary when using 3D image data. This is due to the fixed correspondence that results from the representation of the data as 3D pixels. 

For 2D images, the first algorithm to describe this approach to alignment was the Lucas-Kanade (LK) algorithm \cite{RefWorks:71}. Numerous extensions to the LK algorithm have been proposed \cite{RefWorks:53,RefWorks:72,RefWorks:67} and most are based on \ltwo norm minimisation \cite{RefWorks:72,RefWorks:10,RefWorks:59,RefWorks:73}. Most notably, the inverse compositional framework proposed by Baker and Matthews \cite{RefWorks:74,RefWorks:10} provides a computationally efficient framework for solving the least squares problem. By avoiding the re-computation of the Hessian, the complexity is reduced from $O(n^2 N + n^3)$ to $O(n N + n^3)$ \cite{RefWorks:10} at each iteration, where $n$ is the number of warp parameters and $N$ is the number of pixels in the template image. Baker and Matthews also describe an extension of the LK algorithm into 3D \cite{RefWorks:75} for which an inverse compositional form still exists.

In this work, we follow similar research lines to \cite{RefWorks:6} and perform 3D image alignment using 3D normals. We present two separate alignment methods that seek to use orientation information to align images. Firstly, we study the distribution of the inner product of normals and motivate 3D alignment based on maximisation of the cosine of the angle between the normals. In the second approach, we study the spherical parameterisation of the normals, and motivate alignment based on the maximisation of the sum of the cosines of the azimuth and elevation differences. The robustness of this sum of cosines formulation is shown in \cite{RefWorks:68} and relies on the fact that local mismatches caused by outliers are described by a uniform distribution. Due to this, the areas corrupted by outliers result in approximately zero correlation and thus do not bias the object alignment.

Our algorithms are shown to be robust to corruption caused by simulated bias fields and occlusions. Bias fields occur naturally in magnetic resonance imaging (MRI) due to inhomogeneities in the magnetic field of the MRI machine. They are defined as an undesirable, low frequency signal that blurs an image. A lot of previous work concentrates on the estimation or removal of bias field corruption \cite{RefWorks:82, RefWorks:83}, but not on being robust to the effects of them.

We provide two algorithms that are as efficient as \ltwo norm-based algorithms and are both implemented within the inverse-compositional framework. We use data from the Visible Human data set \cite{RefWorks:81} and assess the performance of the algorithms within a framework similar to that described in \cite{RefWorks:10}.

To summarise, our contributions in this paper are as follows:

\begin{itemize}
    \item We propose the maximisation of gradient correlations as a new cost function for 3D image alignment
    \item We provide two distinct methods to calculate the cosine of the angle between two normals in 3D. The first method is defined as the cosine formed from the inner product of the normals, and the second is formed by transforming the normals into spherical coordinates.
    \item We compare our algorithm to extensions of state-of-the-art 2D alignment algorithms into 3D. The algorithms used for comparison are: enhanced correlation coefficient maximisation \cite{RefWorks:59}, Gabor-Fourier alignment \cite{RefWorks:73}, 3D Lukas-Kanade \cite{RefWorks:75} and iteratively re-weighted least squares \cite{RefWorks:76}.
\end{itemize}